\documentclass[crop=false]{standalone}
\usepackage[utf8]{inputenc}
\usepackage{amsmath}
\usepackage[dvipsnames]{xcolor}
\usepackage{pdfpages}
\usepackage{enumerate}
\usepackage{amssymb}
\usepackage[framemethod=default]{mdframed}
\usepackage[nomarginpar,left=2cm,right=2cm,top = 2cm, bottom = 2cm]{geometry}

\renewcommand{\thesubsection}{\thesection.\alph{subsection}}
\renewcommand{\thesubsubsection}{\thesection.\alph{subsection}.\roman{subsubsection}}

\mdfdefinestyle{theoremstyle}{%
linecolor=black,linewidth=.3pt,%
frametitlerule=true,%
frametitlebackgroundcolor=blue!5,
innertopmargin=\topskip,nobreak=false,
}

\mdfdefinestyle{style2}{frametitle={},%
             linewidth=.3pt,topline=true,backgroundcolor=blue!3!green!8!}

\mdtheorem[style=theoremstyle]{task}{Angabe}

\newmdenv[style = style2,title=false]{solution}

\begin{document}
\begin{task}
Betrachtet wird das beobachtbare System
\[ 
\begin{aligned} \dot{\mathbf{x}} &=\left[\begin{array}{ll}{2} & {1} \\ {3} & {2}\end{array}\right] \mathbf{x}+\left[\begin{array}{l}{1} \\ {0}\end{array}\right] u \\ y &=\left[\begin{array}{ll}{1} & {-1}\end{array}\right] \mathbf{x} \end{aligned}
 \]

 \begin{enumerate}[i]
     \item Ermitteln Sie jene Zustandstransformation $\mathbf{z}=\mathbf{T}_{A} \mathbf{x}$ oder $\mathbf{x}=\mathbf{T}_B \mathbf{Z}$,
die das System in Beobachtbarkeitsnormalform überführt.\\
\emph{Hinweis: Die Transformation muss nicht durchgeführt werden.}
     \begin{solution}
     \[ \mathbf{T}_{B} = \left[ \hat{\mathbf{t}}_1 \ \mathbf{A}\hat{\mathbf{t}}_1 \cdots \mathbf{A}^{n-1} \hat{\mathbf{t}}_1 \right] \quad \text{mit} \quad \mathbf{e}_{n}=\mathbf{M}_{O} \hat{\mathbf{t}}_{1} \]
      \[ \mathbf{M}_{\mathcal{O}} = \begin{pmatrix}
 1 & -1 \\ 
 -1 & -1
 \end{pmatrix},\quad \begin{pmatrix}
 0 \\ 1
 \end{pmatrix} =  \begin{pmatrix}
 1 & -1 \\
 -1 & -1
 \end{pmatrix}
 \begin{pmatrix}
 -\frac{1}{2} \\  -\frac{1}{2}
 \end{pmatrix}\]
 \[ \mathbf{T}_B = \begin{pmatrix}
 -\frac{1}{2} & -\frac{3}{2}\\
 -\frac{1}{2} & -\frac{5}{2}
 \end{pmatrix}\]
     \end{solution}
     \item Es soll ein vollständiger Beobachter so entworfen werden, dass die
Eigenwerte der Beobachterfehlerdynamik bei $-2$ und $-3$ liegen. Ermitteln Sie dafür
die entsprechende Beobachterverstärkung $\hat{\mathbf{k}}$ mit Hilfe der Formel von Ackermann.
 \begin{solution}
 Das geforderte Polynom lautet:
 \[(\lambda +2)(\lambda +3) = \lambda^2 + 5 \lambda + 6\]
 Damit ergibt sich $\hat{\mathbf{k}}$ zu:
 \[ \hat{\mathbf{k}} = -6 \begin{pmatrix}-\frac{1}{2} \\ -\frac{1}{2}\end{pmatrix} - 5 \begin{pmatrix}2 & 1\\3 & 2\end{pmatrix}\begin{pmatrix}-\frac{1}{2} \\ -\frac{1}{2}\end{pmatrix} - \begin{pmatrix}2 & 1\\3 & 2\end{pmatrix}^2\begin{pmatrix}-\frac{1}{2} \\ -\frac{1}{2}\end{pmatrix} = \begin{pmatrix}16 \\ 25\end{pmatrix}\]
 \end{solution}
 \item Der zuvor entworfene vollständige Beobachter liefert den Schätzwert
$\hat{\mathbf{x}}$ für die Zustandsrückführung $u=[-6-4] \hat{\mathbf{x}}$ . Geben Sie alle Eigenwerte des
geschlossenen Kreises an.
\begin{solution}
Die Formel für den vollständigen Beobachter lautet
 \[ 
\begin{array}{l}{\dot{\hat{\mathbf{x}}}=\left(\mathbf{A}+\hat{\mathbf{k}} \mathbf{c}^{T}\right) \hat{\mathbf{x}}+(\mathbf{b}+\hat{\mathbf{k}} d) u-\hat{\mathbf{k}} y} \\ {\hat{\mathbf{x}}=\mathbf{E} \hat{\mathbf{x}}}\end{array}
 \]
 mit $u=[-6 -4] \hat{\mathbf{x}}$ und $y = \mathbf{c}^T \mathbf{x}$ wird der vollständige Beobachter zu:
 \[ 
\begin{array}{l}{\dot{\hat{\mathbf{x}}}=\left(\mathbf{A}+\hat{\mathbf{k}}\mathbf{c}^T+(\mathbf{b}+\hat{\mathbf{k}} d) [-6 \ -4] \right) \hat{\mathbf{x}}-\hat{\mathbf{k}} \mathbf{c}^T \mathbf{x}} \\ {\hat{\mathbf{x}}=\mathbf{E} \hat{\mathbf{x}}}\end{array}
 \]
 Die Strecke selbst lautet:
  \[ 
\begin{array}{l}\dot{\mathbf{x}}= \mathbf{A} \mathbf{x}+\mathbf{b} u\\ y=\mathbf{c}^T \mathbf{x}\end{array}
 \]
 mit $u=[-6-4] \hat{\mathbf{x}}$ wird daraus:
  \[ 
\begin{array}{l}\dot{\mathbf{x}}= \mathbf{A} \mathbf{x}+\mathbf{b}[-6-4] \hat{\mathbf{x}}\\ y=\mathbf{c}^T \mathbf{x}\end{array}
 \]
 Das System kann nun als geschlossenes System geschrieben werden:
   \[ 
\begin{array}{l}\begin{pmatrix} \dot{\mathbf{x}} \\ \dot{\hat{\mathbf{x}}} \end{pmatrix}= \begin{pmatrix} \mathbf{A} & \mathbf{b}[-6-4]\\ -\hat{\mathbf{k}} \mathbf{c}^T & \mathbf{A}+\hat{\mathbf{k}}\mathbf{c}^T+(\mathbf{b}+\hat{\mathbf{k}} d) [-6 \ -4] \end{pmatrix} \begin{pmatrix} \mathbf{x} \\ \hat{\mathbf{x}} \end{pmatrix}\\ y=\begin{pmatrix} \mathbf{c}^T & \mathbf{0} \end{pmatrix} \begin{pmatrix}\mathbf{x} \\ \hat{\mathbf{x}} \end{pmatrix}\end{array}
 \]
 
 Die Dynamikmatrix dieses Systems lautet also explizit:
 \[ \overline{\mathbf{A}} = \begin{pmatrix}2 & 1 & -6 & -4\\ 3 & 2 & 0 & 0\\ -16 & 16 & 12 & -19 \\ -25 & 25 & 28 & 23\end{pmatrix}\]
 Man kann jetzt die Eigenwerte dieser Matrix ausrechnen oder die Eigenwerte des Beobachters (-2 , -3) und die der stabilisierten Strecke seperat ausrechnen, das Ergebnis ist ident. Eigenwerte der geregelten Strecke:
 
 \[ \det \left(\mathbf{A} + \mathbf{b}\begin{pmatrix}-6 -4\end{pmatrix}-\mathbf{E}\lambda\right) = \det\begin{pmatrix}-4-\lambda & -3\\3 & 2-\lambda \end{pmatrix} = \lambda^2 + 2\lambda +1\]
 Die Eigenwerte der stabilisierten Strecke lauten (-1, -1). Alle vier Eigenwerte des geschlossenen Kreises lauten also $-2,-3,-1,-1$.
\end{solution}
 \end{enumerate}
\end{task}
\end{document}