\documentclass[crop=false]{standalone}
\usepackage[utf8]{inputenc}
\usepackage{amsmath}
\usepackage[dvipsnames]{xcolor}
\usepackage{pdfpages}
\usepackage{enumerate}
\usepackage{amssymb}
\usepackage[framemethod=default]{mdframed}
\usepackage[nomarginpar,left=2cm,right=2cm,top = 2cm, bottom = 2cm]{geometry}

\renewcommand{\thesubsection}{\thesection.\alph{subsection}}
\renewcommand{\thesubsubsection}{\thesection.\alph{subsection}.\roman{subsubsection}}

\mdfdefinestyle{theoremstyle}{%
linecolor=black,linewidth=.3pt,%
frametitlerule=true,%
frametitlebackgroundcolor=blue!5,
innertopmargin=\topskip,nobreak=true,
}

\mdfdefinestyle{style2}{frametitle={},%
             linewidth=.3pt,topline=true,backgroundcolor=blue!3!green!8!}

\mdtheorem[style=theoremstyle]{task}{Angabe}

\newmdenv[style = style2,title=false]{solution}

\begin{document}\begin{task}[Erreichbarkeit und Zustandsregler]
Gegeben ist das LTI-System

\[ 
\dot{\mathbf{x}}=\left[\begin{array}{cc}{0} & {1} \\ {5} & {-2}\end{array}\right] \mathbf{x}+\left[\begin{array}{l}{0} \\ {1}\end{array}\right] u
 \]
 
 \begin{enumerate}[i.]
  \item Bestimmen Sie einen Zustandsregler $u=\mathbf{k}^{T} \mathbf{x}$ so, dass die Eigenwerte des geschlossenen Kreises bei $\{-1,-2\}$ liegen.
\begin{solution}
\[\mathbf{M}_R = \left[\mathbf{b}, \mathbf{Ab} \right] = \begin{pmatrix} 0 & 1\\ 1 & -2 \end{pmatrix}\]

Bestimmung von $\mathbf{t}_1^T$:

\[ \begin{pmatrix} 0 & 1 \end{pmatrix} = \mathbf{t}_1^T \begin{pmatrix} 0 & 1\\ 1 & -2 \end{pmatrix} \rightarrow t_1^T = \begin{pmatrix} 1 & 0 \end{pmatrix}\]

Das gewünschte charakteristische Polynom mit seinen Koeffizienten lautet:
\[ p(s) = (\lambda + 1)(\lambda + 2) = \lambda^2 + 3 \lambda + 2 \rightarrow \alpha_0 = 2, \ \alpha_1 = 3\]
Damit ergibt sich der Zustandsregler nach Ackermann zu:
\[ \mathbf{k}^T = -\alpha_0 \mathbf{t}_1^T - \alpha_1 \mathbf{t}_1^T \mathbf{A} - \mathbf{t}_1^T \mathbf{A}^2 = \begin{pmatrix} -7 & -1 \end{pmatrix}\]

Probe:
\[ \det (\mathbf{A} + \mathbf{b}\mathbf{k}^T - \mathbf{E}\lambda) = 
\det \begin{pmatrix} -\lambda & 1 \\ -2 & -3-\lambda \end{pmatrix} = \lambda^2 + 3 \lambda + 2\]
Entspricht dem gewünschten charakteristischen Polynom, Ergebnis stimmt also.
\end{solution}
  \item Geben Sie den Grenzwert $\lim _{t \rightarrow \infty} \mathbf{x}(t)$ für den Anfangszustand $\mathbf{x}(0)=\begin{pmatrix}2 & 1\end{pmatrix}^{T}$ für den geschlossen Kreis an.
\begin{solution}
Im geschlossenen Kreis handelt es sich um ein stabiles System mit negativen Eigenwerten. Das System tendiert nach unendlich langer Zeit zum Zustand $\mathbf{x} = \mathbf{0}$.\\
$\lim _{t \rightarrow \infty} x_{1}(t) = 0$
\end{solution}
  \item Beantworten Sie folgende Aussage mit WAHR oder FALSCH und begründen Sie ausführlich: Ein stabilisierbares System ist automatisch immer steuerbar.
\begin{solution}
Diese Aussage ist Falsch. Es kann ein nicht steuerbares Teilsystem existieren, dass jedoch stabile Eigenwerte besitzt.
\end{solution}
\end{enumerate}
\end{task}
\end{document}